Design objectives:
\begin{itemize}
    \item Load reductions in realistic operating conditions.
    \item Control actions within the limitations of the pitching actuator.
    \item Implementable in discrete time.
\end{itemize}
\subsection{Blade Model}
\subsubsection{Azimuth Based Control}
\subsection{Filter Design Using Zero-Pole Placement}
\subsubsection{Second Order Low Pass Filter}
\subsubsection{Lead Compensator}
A lead compensator is a component of a control system which compensates for undesired lag in a system. A lead compensator is an alternative method to PID control methodology in achieving a desired frequency response of a controller. The transfer function of a lead compensator consists of a pole-zero pair on the real axis of the s-plane:

$$C(s) = K\frac{s-z}{s-p}$$
where $z$, $p$ are the zero and pole of the compensator. For a lead compensator, $|z| < |p|$. The amount of phase lead and the frequency at which the compensation occurs can be specified by the choice of $z$ and $p$. It is convenient to reformulate the compensator transfer function in terms of a time constant, $T$, and attenuation constant, $a$, by substituting $p = 1/T$, $z = 1/aT$, $K=a$, which leads to:
$$C(s) = \frac{1+aTs}{1+Ts}$$
The maximum amount of phase lead, $\phi_m$, occurs at a frequency, $\omega_m$, and are related to $T$ and $a$ by the relations:

$$\sin\phi_m = \frac{a-1}{a+1}$$
$$\omega_m = \frac{1}{T\sqrt{a}}$$
The value of using a lead compensator in the tip deflection controller is that an arbitrary phase lead angle can be set at a target frequency to help overcome the large amount of lag in the system. Given a desired phase lead at a given frequency, $T$ and $a$ can be found, from which the continuous transfer function in terms of $z$, $p$ and $K$ can be found. 
\subsection{Bandpass Filter ?}

\subsection{Controller Discretisation}
Although control design is performed in continuous time, the controller for this application is implemented digitally. For this reason, the continuous controller must be transformed to be implemented as a discrete time controller. The advantage of designing an LTI controller is that there is a large body of knowledge for performing continuous to discrete transformations. 
\\~\\
It is not possible to produce a discrete system which perfectly matches the frequency and time domain performance of its continuous counterpart. For this reason, many methods exist. The Impulse invariant method, zero order hold method, and first order hold method are examples if continuous-to-discrete methods which produce discrete systems which exactly match the time-domain response if a continuous system, namely the impulse, step, and ramp response respectively. The zero order hold method, for example, is the default method used in MATLAB's \texttt{c2d} function. Despite its common use, time-domain invariant methods do not preserve the frequency domain response of a system. As the frequency response of an IPC controller is of high importance, a different discretisation method is sought after.
\\~\\
Two methods were investigated which better preserve the frequency response of the system: pole-zero mapping and the bilinear transform. Both methods showed comparable results due to the high sampling frequency of the controller compared to the frequencies of the wind turbine system.
\subsubsection{Zero-pole mapping}
Pole-zero mapping involves directly mapping the continuous poles and zeros, $s_i$ from the s-domain to equivalent locations in the z-domain, $z_i$, using the transformation $z_i = e^{s_iT_s}$ where $T_s$ is the sampling time of the discrete system. Expressed another way, a continuous transfer function with poles and zeros, $p_i$ and $z_i$:
$$H_a(s) = K_a\frac{\prod_{i=0}^{N}(s-z_i)}{\prod_{i=0}^{N}(s-p_i)}$$
is transformed to a discrete transfer function:
$$H_d(z) = K_d\frac{\prod_{i=0}^{N}(1-e^{z_iT_s}z^{-1})}{\prod_{i=0}^{N}(1-e^{z_iT_s}z^{-1})}$$
TODO
\subsubsection{Bilinear Transformation}

\subsubsection{1P Control Design}
The first feedback controller to successfuly reduce fatigue loads in this project attenuates disturbances at 1P only. This controller consists of a second order low pass filter with a cutoff frequency set at 1P, as well as two identical lead compensators:

$$C_{1P}(s) = K\underbrace{\frac{1}{s^2 + 2\zeta\omega s + \omega^2}}_\text{Low pass filter}\underbrace{\frac{(1-aTs)^2}{(1-Ts)^2}}_\text{Lead compensator}$$

The parameters of this controller are defined in Table XX. Figure XX shows the bode plot of the controller. Note that the controller magnitude response peaks at 1P and is highly attenuated at all other frequencies. Furthermore, the phase response at 1P has a phase lead of XXdegrees to account for the lag in the system. The full continuous transfer function of this system is:
$$continoustransfer function$$
Using the bilinear transform, the controller is discretised to XX:


$$discrete transfer function$$
\begin{table}[]
\centering
\caption{My caption}
\label{my-label}
\begin{tabular}{r|l}
\textbf{Parameter}           & \textbf{Value} \\ \hline
Damping coefficient, $\zeta$ & 0.05           \\
Cutoff frequency, $\omega$   & 1.005          \\
Time constant, $T$           & 0.184          \\
Attenuation constant, $a$    & 29.11         
\end{tabular}
\end{table}
\subsection{HAWC2 integration}
Implementation of a digital IIR filter for each blade.
pseudocode of controller.