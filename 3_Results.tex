\subsection{Quantifying fatigue loads}
The turbine simulations produce complicated time histories of the loads experienced in the various components of the turbine. In order to quantify and compare the fatigue damage experienced in these components, a load spectrum is calculated. A load spectrum decomposes a complicated stress history into stress cycles of varying amplitude. This is achieved using rainflow counting\textbf{source to matsuishi and endo}.The fatigue loads in the turbine components can be quantified by calculating the equivalent load at 1Hz. The equivalent load is the amplitude of a 1Hz oscillating load which produces the same amount of fatigue damage to a component as a mixed load spectrum. It is a way of comparing different load spectra of the same component. The Equivalent load, $S_{eq}$, is calculated for $N_{eq}$ cycles as follows:

$$S_{eq} = \left(\frac{\sum_i S_iN_i^m}{N_{eq}}\right)^{\frac{1}{m}}$$

Where $S_i$ is the $ith$ load cycle amplitude, $N_i$ is number of full cycles at $S_i$, $m$ is the material Wh\"{o}ler curve exponent of the component in question. In this analysis, a Wh\"{o}ler curve exponent of 4 and 10 is used for steel (tower, rotor shaft, etc) and composite materials (ie. rotor blades) respectively. $N_eq$ is the number of cycles experienced of load $S_{eq}$. For 600 second simulations, which is the case for this analysis, A 1Hz equivalent load requires $N_{eq}=600$. 