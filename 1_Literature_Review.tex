
% %\subsection{Background}
% Since the advent of modern wind turbine design, the capacity and size of wind turbines has increased dramatically. From the first electricity-producing wind turbine with a rotor diameter of 17m, the design trend has shown huge growth in rotor size with a 180m rotor on the Adwen 8MW platform. As the size of wind turbines continues to increase, new challenges arise in terms of rotor aerodynamics, structural design and control. large rotors require slender, lighter blades to maintain aerodynamic efficiency and to combat the increase in manufacturing costs. This comes at the cost of exacerbated fatigue loads as the blades experiences various forces as they sweep around the rotor plane.
% \\~\\
% It is well known in the wind industry that loads induced by gravity loads, tower shadow, wind shear, turbulence and gusts can cause significant damage to the blade roots, rotor shaft and the tower \cite{13_Dolan}. By introducing a larger rotor, not only is the turbine more susceptible to this damage, but new instabilities can also be introduced such as aeroelastic flutter, whirling modes, and dynamic coupling such as tower tilt-yaw and blade flap-torsion coupling \cite{7_Berg}.
% \\~\\
% These problems are typically addressed in the aerodynamic and structural design of the wind turbine to ensure significant damage does not occur by introducing safety factors and increasing blade strength. It is becoming increasingly difficult to both predict the effect of these weaknesses and instabilities as well as effectively eliminate them in the design stage without incurring costs in other aspects of the design, such as increased blade mass. For this reason, the wind industry is focusing more attention on innovative methods for reducing loads. One method, which is the focus of this thesis, is the use of active control methods to actively mitigate fatigue loads in order to extend the lifespan of wind turbine components. In particular, the use of tip deflection sensors in a control system will be investigated.
%\\~\\
Research has been performed on reducing loads in the rotor blades using individual pitch control (IPC). By pitching the blades independently, azimuth-dependent loads such as wind shear and tower shadow effects can be mitigated more effectively than with collective pitch control (CPC) \citep{15_bossanyi}. IPC controllers require feedback from each of the blades to effectively reduce the loads. Typically, strain gauges at the root of the blade are used for this purpose. Strain gauges can be difficult to calibrate for long term operation as they are highly sensitive to external factors such as temperature and humidity, and they can experience long term drifting, as well as difficulty in measuring stresses in anisotropic material \citep{18_Papadopoulos}. The research in this project will look into using tip deflection sensors as an alternative to strain gauges. 
\\~\\
Tip deflection sensors are not commonly cited in literature. Bossanyi briefly mentions the possibility of using accelerometers in the blade tips as an alternative to strain gauges, and also mentions the difficulty of maintaining such sensors due to inaccessibility of the blade tips. \cite{5_Bossanyi} \cite{15_bossanyi}. \citet{7_Berg} and \citet{10_Wilson} use tip deflection sensors in their turbine controller designs, however they focus on active flap control. To the best of the author's knowledge, no research has been performed on IPC using tip deflection sensors instead of strain gauges, making this project innovative and relevant to both industry and academia.
\subsection{Fatigue Load Mitigation in Wind Turbine Control Systems}
The concept of active load control is not a new concept in the wind industry. Tower and drive train loads can successfully be reduced by extending the basic speed control used in modern variable speed wind turbines.
\subsubsection{Torque Control}
Torque control is used to maintain optimal power output when the wind turbine is operating below rated wind speed. The idea is to balance the aerodynamic and generator torque to achieve a maximum power coefficient $C_p$. A common control strategy to achieve this is to use a proportionality relationship between demanded torque $Q$, and the square of the generator angular velocity $\omega_g$
$$Q = \frac{\pi \rho R^5 C_p}{2\lambda^3 G^3}\omega_g^2$$ \cite{15_bossanyi}

torque control has also been used to introduce side-side tower damping and mitigate drive train torsion vibration....

\subsection{Sensing methods}
Real time measurements have become increasingly important for wind turbines and wind farms over recent years, allowing for sophisticated digital control. The stream of data is typically referred to as Supervisory control and data acquisition (SCADA), and can comprise of a variety of novel measurements from components of the turbine. A summary of sensors relevant to active load reduction is described in this section.
\subsubsection{Rotor Azimuth measurement}
Modern wind turbines use rotor angular velocity for the power control loop rather than the rotor azimuth angle. For IPC, a measurement of the rotor azimuth angle is vital in transforming the rotor frame of reference to the fixed tower frame of reference. This is because many load oscillations are dependent on the azimuth angle, such as tower shadow, wind shear and yaw misalignment. 
\\~\\
\citet{4_trudnowski} demonstrated effective control of flap wise loads given only the rotor azimuth angle. Given a precise model of how the wind field varies with rotor angle, for example, \citet{13_Dolan}, this type of control can be possible. However,the wind field can vary dramatically for a given site, and therefore a controller using only rotor azimuth angle would likely be ineffective in the real world. Additionally, fluctuations due to turbulence can not be effectively reduced without additional sensing and controller feedback. Nevertheless, the use of rotor azimuth sensing in conjunction with additional sensors is relevant to IPC, specifically for Coleman-transform based control.

\subsubsection{Wind measurements}
Cup anemometers and wind vane are the standard, IEC-approved method for taking wind speed and direction measurements, however they are unable to capture high frequency components of the wind field as well as spacial variation. As with the power control, load reduction controllers can benefit from the low frequency cup anemometer measurements for gain scheduling. Wind fields can vary greatly over time and space, more advanced sensors may be beneficial to observe how the wind field varies with rotor azimuth angle. LIDAR measurements can provide wind field data at both a high frequency and spacial resolution, and can provide useful information to a control system regarding gusts and field variations before they are experienced by the turbine. \citet{6_Mirzaei} Demonstrated that model predictive control methods were more effective than PI control when used with upstream LIDAR measurements especially in transitioning wind conditions. 
\\~\\
Although the wind measurement techniques is beyond the scope of this project, it should be noted that tip deflection sensors could provide useful information in terms of gust control when used in conjunction with wind speed measurements. \citet{2_Kanev} uses blade root bending moment measurements, both in-plane and out-of-plane, in order to better estimate the blade-effective wind speed, and to additionally detect gust events. An extended Kalman filter observer was used as a nonlinear observer for this study.
\\~\\
...
\subsubsection{Tower Acceleration}
Tower motion, both in the fore-aft and side-side direction, is the source of fatigue loading in the tower. Additionally, the fore-aft dynamics of the tower is coupled with the power output of the turbine due fluctuations in the effective wind speed at the rotor. For this reason, tower damping is often a control objective using pitch and torque control. In order to effectively reduce this oscillation, measurements of the tower acceleration are generally used as feedback to the control system. This can be performed with an accelerometer in the nacelle of the wind turbine \cite{15_bossanyi}.
\subsubsection{Strain Gauges}
Commonly used in literature for IPC. No apparent reason for using strain gauges over tip deflection sensors. will write more here.
\subsubsection{Tip Deflection Sensors}
Tip deflection sensors are not commonly cited in literature. Bossanyi briefly mentions the possibility of using accelerometers in the blade tips as an alternative to strain gauges, and also mentions the difficulty of maintaining such sensors due to inaccessibility of the blade tips. \cite{5_Bossanyi} \cite{15_bossanyi}. \citet{7_Berg} and \citet{10_Wilson} use tip deflection sensors in their turbine controller designs, however they focus on active flap control. To the best of the author's knowledge, no research has been performed on IPC using tip deflection sensors instead of strain gauges. 
\subsection{Wind Turbine Modelling}
\textcolor{red}{will elaborate on this section}
\subsubsection{Structural}
Something on Euler-Bernoulli beam theory vs Timoshenko, and that HAWC2 uses Timoshenko. Something about the simplified structural models(usually 1 mode for blades and tower) used in literature, and the different assumptions they make and why. Usually a simplified model will suffice and provides transparency to the controller.
\subsubsection{Aerodynamic Modelling}
Aerodynamic models can be complicated (CFD, actuator models), difficult to implement in a control system. Can use advanced models to get linear approximations of system dynamics. Talk about how this is addressed in literature for IPC.
\cite{11_Wang}

\subsection{Observer design}
Kalman filter \cite{15_bossanyi}, extended kalman filter \cite{2_Kanev}. Implementation, and inclusion of colored wind spectrum \cite{14_Selvam}. 
\\
A fundamental aspect of a wind turbine controller is the state observer. The state observer recreates the structural and aerodynamic state of the wind turbine system. A good observer is able to improve controller response by providing a short term prediction how the state will change. It is able to track the state precisely over different working conditions, and in the presence of noise, both from the measurements and from the system dynamics itself. A standard and highly effective observer design is typically the Kalman filter. In linear control theory, a Kalman filter is straight forward to design if the process and measurement noise can be known a priori. The Kalman filter can be extended to nonlinear systems by using an extended Kalman filter, which uses a linearized approximation of the current operating point. Something about the unscented Kalman filter. Something about a second order extended Kalman filter (augmented Kalman filter) (maybe not?).
\\~\\
An alternative and novel approach to state observation is using machine learning algorithms. Elaborate on this?
\subsection{IPC Controller design}
As the control problem at hand is nonlinear, there are a number of controller designs which have been proposed in literature regarding IPC. One common step is performing a transformation to convert the system into a time invariant system using the Coleman transformation, which is a special case of the the multi-blade coordinate transformation, and also known as the d-q transformation borrowed from electrical machine theory. The Coleman transformation removes the dependency of rotor azimuth angle from the system, allowing for linear control methods to be used. More specifically, the Coleman transformation converts the flapwise and edgewise loads into tower tilt and yaw moment, effectively representing 1P loads in the rotating frame into 0P and 2P loads in the fixed frame. In addition to the Coleman transformation, many control ideologies have been adopted in literature. Some of the more common methods are outlined in the following section.

\subsubsection{PI Control}
PI control is the most basic control method used in literature. PI control is a linear control method which provides a control signal proportional to to the error signal and its integral over time. When used in IPC, it is assumed that the yaw and tilt moment decomposition resulting from the Coleman transformation are decoupled. This allows the tilt and yaw to be treated independently with two separate PI controllers. PI control is often used as a benchmark for testing other control methods in IPC. Despite the simplicity of the control method, it shows comparable results to more advanced models at low bandwidths and is more transparent in its implementation \cite{6_Mirzaei}, \cite{14_Selvam}. 
\\~\\
The underlying assumption of using PI control for IPC is the decoupling between the tilt and yaw degrees of freedom. In reality, this is not the case. \citet{1_Lu} presents the weaknesses of the decoupling assumption. Additionally, performing the Coleman transformation and its inverse causes a frequency shift in the system dynamics which should be taken into account in the controller design. 
\subsubsection{Linear-Quadratic-Gaussian Regulator}
A linear-quadratic-Gaussian regulator (LQG) is a multi-input-multi-output (MIMO) control method that provides optimal control to uncertain linear system with white noise disturbances. A LQG is a combination of a linear-quadratic regulator (LQR) with a Kalman filter as a state estimator. For a linear system, the LQR and the Kalman filter can be designed independently of each other. The controller is able to act on multiple sensor inputs and achieve multiple control objectives by changing the weights of a cost function.  \citet{14_Selvam} and \citet{5_Bossanyi} use LQG, and it works pretty well (elaborate more).
\\~\\
The weakness of LQG is that it is a linear controller, which means that it may not perform well on the nonlinear dynamics of a wind turbine. This can usually be addressed by linearising the system about control points and with gain scheduling.
\subsubsection{Loop Shaping}
Loop shaping is a control technique conducted in the frequency domain, where the frequency response of a close loop system can be adjusted to achieve certain performance and robustness objectives. Used in \cite{1_Lu} to overcome the tilt-yaw coupling problem in PI control. Used in \citet{17_Geyler} on a simple wind turbine model. The unmodelled behaviour is taken into account in the robustness of the \hinfty controller. Also successfully used in \citet{2_Kanev}.
\subsubsection{Model Predictive Control}
Model predictive control used in \cite{6_Mirzaei}.
\subsubsection{Machine Learning}

Neural networks and machine learning. \cite{8_Wang} \cite{3_Treiber}. bad because black box, and does not add any benefits over modern control theory. the motivation is lacking according to \citet{15_bossanyi}

...Cyclic pitch control - Selvam.



% The unique aspect of this paper is in the use of tip deflection sensors instead of a strain gauge at the blade roots. This has rarely been investigated in literature. \citet{7_Berg} addresses the use of tip deflection readings as an input for a turbine control system using active flaps. DESCRIBE RESULTS. Although active flaps can provide a high frequency response for a turbine controller, they have yet to be used in commercial turbine models, and are typically reserved for academic research. Active flaps are expected to increase operating and maintenance costs due to the additional moving parts required. XX mentions the possibility of using tip deflection readings as an alternative input. Apart from these references, tip deflection sensors have not been largely considered in literature to the best of the author's knowledge. It follows that the use of tip deflection sensors in conjunction with IPC has not been explored, and will remain the focus of this paper.
% \\~\\
% \textcolor{red}{Not sure where to write a section on the types of loads, or what to cite.
% Two types of loads are typically considered in engineering design: fatigue load and ultimate load. Fatigue loads occur as a result of structural oscillations over a long period of time. Calculations relating to fatigue largely consider the magnitude of the loading rather than the frequency. The rainflow counting algorithm developed by XX, is commonplace in engineering in distinguishing the the number of cycles of different magnitudes in a timeseries. Using the rainflow counting results, a short term equivalent load can be determined using BLAH BLAH. An equivalent load is a way of representing the fatigue load, which may consist of various amplitude oscillations at different frequencies, as a single sinusoidal oscillation with a fixed frequency (typically 1Hz). One control objective could be to minimise the short term equivalent load of the wind turbine blades over the operating lifetime.
% \\~\\ 
% The second type of load that is considered is the ultimate load, which is the largest expected load to occur over a short period of time. Extreme loads which exceed the ultimate strength of a component can lead to instantaneous catastrophic failure of a wind turbine, and can occur due to extreme winds or gusts. IEC standards consider the most extreme load within a 10 minute period which is expected to occur over 50 year time period. Extreme analysis, such as the use of a Gumbel distribution is commonly used to predict the 50 year extreme load based on limited time series data.}
\\~\\
% \textcolor{red}{Take into account torsion?
% \citet{7_Berg} says flap bending - torsion coupling increases with rotor sizze. ignores torsion as impact is minor. }
% \\\textcolor{red}{Is cyclic pitch control the same as IPC? Should I make a distinction?}
% \\
% \textcolor{red}{Is a sensor to measure azimuth angle the same as the sensor used to measure rotor speed? }\\
% \textcolor{red}{What are the benefits of using tip deflection sensors}